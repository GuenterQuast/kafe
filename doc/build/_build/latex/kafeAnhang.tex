% Generated by Sphinx.
\def\sphinxdocclass{report}
\documentclass[a4paper,10pt,english]{sphinxmanual}
\usepackage[utf8]{inputenc}
\DeclareUnicodeCharacter{00A0}{\nobreakspace}
\usepackage[T1]{fontenc}
\usepackage{babel}
% Euler for math | Palatino for rm | Helvetica for ss
\renewcommand{\rmdefault}{ppl} % rm
\linespread{1.05}        % Palatino needs more leading
\usepackage[scaled]{helvet} % ss
\renewcommand{\ttdefault}{lmtt} %tt
\usepackage[sc,osf]{mathpazo} % math
\usepackage[euler-digits]{eulervm}

\usepackage[Bjarne]{fncychap}
\usepackage{longtable}
\usepackage{sphinx}
\usepackage{multirow}


\title{kafe Documentation}
\date{August 21, 2013}
\release{0.3alpha46}
\author{Daniel Savoiu}
\newcommand{\sphinxlogo}{}
\renewcommand{\releasename}{Release}
\makeindex

\makeatletter
\def\PYG@reset{\let\PYG@it=\relax \let\PYG@bf=\relax%
    \let\PYG@ul=\relax \let\PYG@tc=\relax%
    \let\PYG@bc=\relax \let\PYG@ff=\relax}
\def\PYG@tok#1{\csname PYG@tok@#1\endcsname}
\def\PYG@toks#1+{\ifx\relax#1\empty\else%
    \PYG@tok{#1}\expandafter\PYG@toks\fi}
\def\PYG@do#1{\PYG@bc{\PYG@tc{\PYG@ul{%
    \PYG@it{\PYG@bf{\PYG@ff{#1}}}}}}}
\def\PYG#1#2{\PYG@reset\PYG@toks#1+\relax+\PYG@do{#2}}

\def\PYG@tok@gd{\def\PYG@tc##1{\textcolor[rgb]{0.63,0.00,0.00}{##1}}}
\def\PYG@tok@gu{\let\PYG@bf=\textbf\def\PYG@tc##1{\textcolor[rgb]{0.50,0.00,0.50}{##1}}}
\def\PYG@tok@gt{\def\PYG@tc##1{\textcolor[rgb]{0.00,0.25,0.82}{##1}}}
\def\PYG@tok@gs{\let\PYG@bf=\textbf}
\def\PYG@tok@gr{\def\PYG@tc##1{\textcolor[rgb]{1.00,0.00,0.00}{##1}}}
\def\PYG@tok@cm{\let\PYG@it=\textit\def\PYG@tc##1{\textcolor[rgb]{0.25,0.50,0.56}{##1}}}
\def\PYG@tok@vg{\def\PYG@tc##1{\textcolor[rgb]{0.73,0.38,0.84}{##1}}}
\def\PYG@tok@m{\def\PYG@tc##1{\textcolor[rgb]{0.13,0.50,0.31}{##1}}}
\def\PYG@tok@mh{\def\PYG@tc##1{\textcolor[rgb]{0.13,0.50,0.31}{##1}}}
\def\PYG@tok@cs{\def\PYG@tc##1{\textcolor[rgb]{0.25,0.50,0.56}{##1}}\def\PYG@bc##1{\colorbox[rgb]{1.00,0.94,0.94}{##1}}}
\def\PYG@tok@ge{\let\PYG@it=\textit}
\def\PYG@tok@vc{\def\PYG@tc##1{\textcolor[rgb]{0.73,0.38,0.84}{##1}}}
\def\PYG@tok@il{\def\PYG@tc##1{\textcolor[rgb]{0.13,0.50,0.31}{##1}}}
\def\PYG@tok@go{\def\PYG@tc##1{\textcolor[rgb]{0.19,0.19,0.19}{##1}}}
\def\PYG@tok@cp{\def\PYG@tc##1{\textcolor[rgb]{0.00,0.44,0.13}{##1}}}
\def\PYG@tok@gi{\def\PYG@tc##1{\textcolor[rgb]{0.00,0.63,0.00}{##1}}}
\def\PYG@tok@gh{\let\PYG@bf=\textbf\def\PYG@tc##1{\textcolor[rgb]{0.00,0.00,0.50}{##1}}}
\def\PYG@tok@ni{\let\PYG@bf=\textbf\def\PYG@tc##1{\textcolor[rgb]{0.84,0.33,0.22}{##1}}}
\def\PYG@tok@nl{\let\PYG@bf=\textbf\def\PYG@tc##1{\textcolor[rgb]{0.00,0.13,0.44}{##1}}}
\def\PYG@tok@nn{\let\PYG@bf=\textbf\def\PYG@tc##1{\textcolor[rgb]{0.05,0.52,0.71}{##1}}}
\def\PYG@tok@no{\def\PYG@tc##1{\textcolor[rgb]{0.38,0.68,0.84}{##1}}}
\def\PYG@tok@na{\def\PYG@tc##1{\textcolor[rgb]{0.25,0.44,0.63}{##1}}}
\def\PYG@tok@nb{\def\PYG@tc##1{\textcolor[rgb]{0.00,0.44,0.13}{##1}}}
\def\PYG@tok@nc{\let\PYG@bf=\textbf\def\PYG@tc##1{\textcolor[rgb]{0.05,0.52,0.71}{##1}}}
\def\PYG@tok@nd{\let\PYG@bf=\textbf\def\PYG@tc##1{\textcolor[rgb]{0.33,0.33,0.33}{##1}}}
\def\PYG@tok@ne{\def\PYG@tc##1{\textcolor[rgb]{0.00,0.44,0.13}{##1}}}
\def\PYG@tok@nf{\def\PYG@tc##1{\textcolor[rgb]{0.02,0.16,0.49}{##1}}}
\def\PYG@tok@si{\let\PYG@it=\textit\def\PYG@tc##1{\textcolor[rgb]{0.44,0.63,0.82}{##1}}}
\def\PYG@tok@s2{\def\PYG@tc##1{\textcolor[rgb]{0.25,0.44,0.63}{##1}}}
\def\PYG@tok@vi{\def\PYG@tc##1{\textcolor[rgb]{0.73,0.38,0.84}{##1}}}
\def\PYG@tok@nt{\let\PYG@bf=\textbf\def\PYG@tc##1{\textcolor[rgb]{0.02,0.16,0.45}{##1}}}
\def\PYG@tok@nv{\def\PYG@tc##1{\textcolor[rgb]{0.73,0.38,0.84}{##1}}}
\def\PYG@tok@s1{\def\PYG@tc##1{\textcolor[rgb]{0.25,0.44,0.63}{##1}}}
\def\PYG@tok@gp{\let\PYG@bf=\textbf\def\PYG@tc##1{\textcolor[rgb]{0.78,0.36,0.04}{##1}}}
\def\PYG@tok@sh{\def\PYG@tc##1{\textcolor[rgb]{0.25,0.44,0.63}{##1}}}
\def\PYG@tok@ow{\let\PYG@bf=\textbf\def\PYG@tc##1{\textcolor[rgb]{0.00,0.44,0.13}{##1}}}
\def\PYG@tok@sx{\def\PYG@tc##1{\textcolor[rgb]{0.78,0.36,0.04}{##1}}}
\def\PYG@tok@bp{\def\PYG@tc##1{\textcolor[rgb]{0.00,0.44,0.13}{##1}}}
\def\PYG@tok@c1{\let\PYG@it=\textit\def\PYG@tc##1{\textcolor[rgb]{0.25,0.50,0.56}{##1}}}
\def\PYG@tok@kc{\let\PYG@bf=\textbf\def\PYG@tc##1{\textcolor[rgb]{0.00,0.44,0.13}{##1}}}
\def\PYG@tok@c{\let\PYG@it=\textit\def\PYG@tc##1{\textcolor[rgb]{0.25,0.50,0.56}{##1}}}
\def\PYG@tok@mf{\def\PYG@tc##1{\textcolor[rgb]{0.13,0.50,0.31}{##1}}}
\def\PYG@tok@err{\def\PYG@bc##1{\fcolorbox[rgb]{1.00,0.00,0.00}{1,1,1}{##1}}}
\def\PYG@tok@kd{\let\PYG@bf=\textbf\def\PYG@tc##1{\textcolor[rgb]{0.00,0.44,0.13}{##1}}}
\def\PYG@tok@ss{\def\PYG@tc##1{\textcolor[rgb]{0.32,0.47,0.09}{##1}}}
\def\PYG@tok@sr{\def\PYG@tc##1{\textcolor[rgb]{0.14,0.33,0.53}{##1}}}
\def\PYG@tok@mo{\def\PYG@tc##1{\textcolor[rgb]{0.13,0.50,0.31}{##1}}}
\def\PYG@tok@mi{\def\PYG@tc##1{\textcolor[rgb]{0.13,0.50,0.31}{##1}}}
\def\PYG@tok@kn{\let\PYG@bf=\textbf\def\PYG@tc##1{\textcolor[rgb]{0.00,0.44,0.13}{##1}}}
\def\PYG@tok@o{\def\PYG@tc##1{\textcolor[rgb]{0.40,0.40,0.40}{##1}}}
\def\PYG@tok@kr{\let\PYG@bf=\textbf\def\PYG@tc##1{\textcolor[rgb]{0.00,0.44,0.13}{##1}}}
\def\PYG@tok@s{\def\PYG@tc##1{\textcolor[rgb]{0.25,0.44,0.63}{##1}}}
\def\PYG@tok@kp{\def\PYG@tc##1{\textcolor[rgb]{0.00,0.44,0.13}{##1}}}
\def\PYG@tok@w{\def\PYG@tc##1{\textcolor[rgb]{0.73,0.73,0.73}{##1}}}
\def\PYG@tok@kt{\def\PYG@tc##1{\textcolor[rgb]{0.56,0.13,0.00}{##1}}}
\def\PYG@tok@sc{\def\PYG@tc##1{\textcolor[rgb]{0.25,0.44,0.63}{##1}}}
\def\PYG@tok@sb{\def\PYG@tc##1{\textcolor[rgb]{0.25,0.44,0.63}{##1}}}
\def\PYG@tok@k{\let\PYG@bf=\textbf\def\PYG@tc##1{\textcolor[rgb]{0.00,0.44,0.13}{##1}}}
\def\PYG@tok@se{\let\PYG@bf=\textbf\def\PYG@tc##1{\textcolor[rgb]{0.25,0.44,0.63}{##1}}}
\def\PYG@tok@sd{\let\PYG@it=\textit\def\PYG@tc##1{\textcolor[rgb]{0.25,0.44,0.63}{##1}}}

\def\PYGZbs{\char`\\}
\def\PYGZus{\char`\_}
\def\PYGZob{\char`\{}
\def\PYGZcb{\char`\}}
\def\PYGZca{\char`\^}
\def\PYGZsh{\char`\#}
\def\PYGZpc{\char`\%}
\def\PYGZdl{\char`\$}
\def\PYGZti{\char`\~}
% for compatibility with earlier versions
\def\PYGZat{@}
\def\PYGZlb{[}
\def\PYGZrb{]}
\makeatother

\begin{document}

\maketitle
\tableofcontents
\phantomsection\label{index::doc}\phantomsection\label{index:module-kafe}\index{kafe (module)}


\textbf{kafe} is a data fitting framework designed for use in
undergraduate physics lab courses. It provides a basic \emph{Python} toolkit
for fitting and plotting using already available \emph{Python} packages
such as \emph{NumPy} and \emph{matplotlib}, as well as \emph{CERN} \emph{ROOT}`s version of
the \emph{Minuit} minimizer.
\setbox0\vbox{
\begin{minipage}{0.95\linewidth}
\textbf{Contents}

\medskip

\begin{itemize}
\item {} 
{\hyperref[index:kafe-karlsruhe-fit-environment-documentation]{\textbf{kafe} -- Karlsruhe Fit Environment documentation}}
\begin{itemize}
\item {} 
{\hyperref[index:summary]{Summary}}

\item {} 
{\hyperref[index:api-documentation]{\emph{API} documentation}}

\end{itemize}

\end{itemize}
\end{minipage}}
\begin{center}\setlength{\fboxsep}{5pt}\shadowbox{\box0}\end{center}


\chapter{Summary}
\label{index:kafe-karlsruhe-fit-environment-documentation}\label{index:summary}
The package provides a simple approach to fitting using variance-covariance
matrices, thus allowing for error correlations to be taken into account. This
implementation's error model assumes the measurement data (dependent variable)
is distributed according to a \emph{Gaussian} distribution centered at its ``true''
value. The spread of the distribution is given as a ($1\sigma$)-error.

An ``errors-in-variables'' model is also implemented to take uncertainties in
the independent variable (\emph{x} errors) into account. This is done by
specifying/constructing a separate variance-covariance matrix for the \emph{x} axis
and ``projecting'' it onto the \emph{y} error matrix. If the fit function
is approximated in each point by its tangent line, the \emph{Gaussian} errors in
the \emph{x} direction are not warped by this projection.

$\ldots$

For examples on how to use \textbf{kafe}, see the \code{examples} folder. Consulting
the {\hyperref[index:api]{API}} can also be helpful.


\chapter{\emph{API} documentation}
\label{index:api-documentation}

\section{kafe Package}
\label{index:api}\label{index:kafe-package}

\subsection{\texttt{kafe} Package}
\label{index:id1}\phantomsection\label{index:module-kafe.__init__}\index{kafe.\_\_init\_\_ (module)}
A Python package for fitting and plotting for use in physics lab courses.

This Python package allows fitting of user-defined functions to data. A dataset is
represented by a \emph{Dataset} object which stores measurement data as \emph{NumPy} arrays.
The uncertainties of the data are also stored in the \emph{Dataset} as an \emph{error matrix},
allowing for both correlated and uncorrelated errors to be accurately represented.

The constructor of a \emph{Dataset} object accepts several keyword arguments and can be used
to construct a \emph{Dataset} out of data which has been loaded into \emph{Python} as \emph{NumPy} arrays.
Alternatively, a plain-text representation of a \emph{Dataset} can be read from a file.

Also provided are helper functions which construct a \emph{Dataset} object from a
file containing column data (one measurement per row, column order can be specified).


\subsection{\texttt{\_version\_info} Module}
\label{index:module-kafe._version_info}\label{index:version-info-module}\index{kafe.\_version\_info (module)}

\subsection{\texttt{constants} Module}
\label{index:constants-module}\label{index:module-kafe.constants}\index{kafe.constants (module)}\phantomsection\label{index:module-constants}\index{constants (module)}\index{F\_SIGNIFICANCE (in module kafe.constants)}

\begin{fulllineitems}
\phantomsection\label{index:kafe.constants.F_SIGNIFICANCE}\pysigline{\bfcode{F\_SIGNIFICANCE}\strong{ = 2}}
Set significance for returning results and errors
N = rounding error to N significant digits and value
to the same order of magnitude as the error.

\end{fulllineitems}

\index{G\_PADDING\_FACTOR\_X (in module kafe.constants)}

\begin{fulllineitems}
\phantomsection\label{index:kafe.constants.G_PADDING_FACTOR_X}\pysigline{\bfcode{G\_PADDING\_FACTOR\_X}\strong{ = 1.2}}
factor by which to expand \emph{x} data range

\end{fulllineitems}

\index{G\_PADDING\_FACTOR\_Y (in module kafe.constants)}

\begin{fulllineitems}
\phantomsection\label{index:kafe.constants.G_PADDING_FACTOR_Y}\pysigline{\bfcode{G\_PADDING\_FACTOR\_Y}\strong{ = 1.2}}
factor by which to expand \emph{y} data range

\end{fulllineitems}

\index{G\_PLOT\_POINTS (in module kafe.constants)}

\begin{fulllineitems}
\phantomsection\label{index:kafe.constants.G_PLOT_POINTS}\pysigline{\bfcode{G\_PLOT\_POINTS}\strong{ = 200}}
number of plot points for plotting the function

\end{fulllineitems}

\index{M\_CONFIDENCE\_LEVEL (in module kafe.constants)}

\begin{fulllineitems}
\phantomsection\label{index:kafe.constants.M_CONFIDENCE_LEVEL}\pysigline{\bfcode{M\_CONFIDENCE\_LEVEL}\strong{ = 0.05}}
Confidence level for hypythesis test. A fit is rejected it $\chi^2_\text{prob}$ is smaller than this constant

\end{fulllineitems}

\index{M\_MAX\_ITERATIONS (in module kafe.constants)}

\begin{fulllineitems}
\phantomsection\label{index:kafe.constants.M_MAX_ITERATIONS}\pysigline{\bfcode{M\_MAX\_ITERATIONS}\strong{ = 6000}}
Maximum \emph{Minuit} iterations until aborting the process

\end{fulllineitems}

\index{M\_MAX\_X\_FIT\_ITERATIONS (in module kafe.constants)}

\begin{fulllineitems}
\phantomsection\label{index:kafe.constants.M_MAX_X_FIT_ITERATIONS}\pysigline{\bfcode{M\_MAX\_X\_FIT\_ITERATIONS}\strong{ = 2}}
Number of maximal additional iterations for \emph{x} fit (0 disregards \emph{x} errors)

\end{fulllineitems}

\index{M\_TOLERANCE (in module kafe.constants)}

\begin{fulllineitems}
\phantomsection\label{index:kafe.constants.M_TOLERANCE}\pysigline{\bfcode{M\_TOLERANCE}\strong{ = 0.1}}
\emph{Minuit} tolerance level

\end{fulllineitems}



\subsection{\texttt{dataset} Module}
\label{index:dataset-module}\label{index:module-kafe.dataset}\index{kafe.dataset (module)}\phantomsection\label{index:module-dataset}\index{dataset (module)}\index{Dataset (class in kafe.dataset)}

\begin{fulllineitems}
\phantomsection\label{index:kafe.dataset.Dataset}\pysiglinewithargsret{\strong{class }\bfcode{Dataset}}{\emph{**kwargs}}{}
The \emph{Dataset} object is a data structure for storing measurement and error data. In this implementation,
the \emph{Dataset} has the compulsory field \emph{data}, which is used for storing the measurement data,
and another field \emph{cov\_mats}, used for storing the covariance matrix for each axis.

There are several ways a \emph{Dataset} can be constructed. The most straightforward way is to specify an
input file containing a plain-text representation of the dataset:

\begin{Verbatim}[commandchars=\\\{\}]
\PYG{g+gp}{\textgreater{}\textgreater{}\textgreater{} }\PYG{n}{my\PYGZus{}dataset} \PYG{o}{=} \PYG{n}{Dataset}\PYG{p}{(}\PYG{n}{input\PYGZus{}file}\PYG{o}{=}\PYG{l+s}{'}\PYG{l+s}{/path/to/file}\PYG{l+s}{'}\PYG{p}{)}
\end{Verbatim}

or

\begin{Verbatim}[commandchars=\\\{\}]
\PYG{g+gp}{\textgreater{}\textgreater{}\textgreater{} }\PYG{n}{my\PYGZus{}dataset} \PYG{o}{=} \PYG{n}{Dataset}\PYG{p}{(}\PYG{n}{input\PYGZus{}file}\PYG{o}{=}\PYG{n}{my\PYGZus{}file\PYGZus{}object}\PYG{p}{)}
\end{Verbatim}

If an \emph{input\_file} keyword is provided, all other input is ignored. The \emph{Dataset} plain-text representation
format is as follows:

\begin{Verbatim}[commandchars=\\\{\}]
\# x data
x\_1  sigma\_x\_1  
x\_2  sigma\_x\_2  cor\_x\_12
...  ...        ...       ...
x\_N  sigma\_x\_N  cor\_x\_1N  ...  cor\_x\_NN

\# y data
y\_1  sigma\_y\_1  
y\_2  sigma\_y\_2  cor\_y\_12
...  ...        ...       ...
y\_N  sigma\_y\_N  cor\_y\_1N  ...  cor\_y\_NN
\end{Verbatim}

Here, the \emph{sigma\_...} represents the statistical error of the data point and \emph{cor\_...\_ij} is the
correlation coefficient between the \emph{i}-th and \emph{j}-th data point.

Alternatively, field data can be set by passing iterables as keyword arguments. Available keywords
for this purpose are:

\textbf{data} : tuple/list of tuples/lists/arrays of floats
\begin{quote}

a tuple/list of measurement data. Each element of the tuple/list must be iterable and 
be of the same length. The first element of the \textbf{data} tuple/list is assumed to be
the \emph{x} data, and the second to be the \emph{y} data:

\begin{Verbatim}[commandchars=\\\{\}]
\PYG{g+gp}{\textgreater{}\textgreater{}\textgreater{} }\PYG{n}{my\PYGZus{}dataset} \PYG{o}{=} \PYG{n}{Dataset}\PYG{p}{(}\PYG{n}{data}\PYG{o}{=}\PYG{p}{(}\PYG{p}{[}\PYG{l+m+mf}{0.}\PYG{p}{,} \PYG{l+m+mf}{1.}\PYG{p}{,} \PYG{l+m+mf}{2.}\PYG{p}{,} \PYG{l+m+mf}{3.}\PYG{p}{,} \PYG{l+m+mf}{4.}\PYG{p}{]}\PYG{p}{,} \PYG{p}{[}\PYG{l+m+mf}{1.23}\PYG{p}{,} \PYG{l+m+mf}{3.45}\PYG{p}{,} \PYG{l+m+mf}{5.62}\PYG{p}{,} \PYG{l+m+mf}{7.88}\PYG{p}{,} \PYG{l+m+mf}{9.64}\PYG{p}{]}\PYG{p}{)}\PYG{p}{)}
\end{Verbatim}

Alternatively, x-y value pairs can also be passed as \textbf{data}. The following is equivalent to the above:

\begin{Verbatim}[commandchars=\\\{\}]
\PYG{g+gp}{\textgreater{}\textgreater{}\textgreater{} }\PYG{n}{my\PYGZus{}dataset} \PYG{o}{=} \PYG{n}{Dataset}\PYG{p}{(}\PYG{n}{data}\PYG{o}{=}\PYG{p}{(}\PYG{p}{[}\PYG{l+m+mf}{0.0}\PYG{p}{,} \PYG{l+m+mf}{1.23}\PYG{p}{]}\PYG{p}{,} \PYG{p}{[}\PYG{l+m+mf}{1.0}\PYG{p}{,} \PYG{l+m+mf}{3.45}\PYG{p}{]}\PYG{p}{,} \PYG{p}{[}\PYG{l+m+mf}{2.0}\PYG{p}{,} \PYG{l+m+mf}{5.62}\PYG{p}{]}\PYG{p}{,} \PYG{p}{[}\PYG{l+m+mf}{3.0}\PYG{p}{,} \PYG{l+m+mf}{7.88}\PYG{p}{]}\PYG{p}{,} \PYG{p}{[}\PYG{l+m+mf}{4.0}\PYG{p}{,} \PYG{l+m+mf}{9.64}\PYG{p}{]}\PYG{p}{)}\PYG{p}{)}
\end{Verbatim}

In case the \emph{Dataset} contains two data points, the ordering is ambiguous. In this case, the
first ordering (\emph{x} data first, then \emph{y} data) is assumed.
\end{quote}

\textbf{cov\_mats} : tuple/list of \emph{numpy.matrix}
\begin{quote}

a tuple/list of two-dimensional iterables containing the covariance matrices for \emph{x} and \emph{y}, in that
order. Covariance matrices can be any sort of two-dimensional NxN iterables, assuming N is the number
of data points.

\begin{Verbatim}[commandchars=\\\{\}]
\PYG{g+gp}{\textgreater{}\textgreater{}\textgreater{} }\PYG{n}{my\PYGZus{}dataset} \PYG{o}{=} \PYG{n}{Dataset}\PYG{p}{(}\PYG{n}{data}\PYG{o}{=}\PYG{p}{(}\PYG{p}{[}\PYG{l+m+mf}{0.}\PYG{p}{,} \PYG{l+m+mf}{1.}\PYG{p}{,} \PYG{l+m+mf}{2.}\PYG{p}{]}\PYG{p}{,} \PYG{p}{[}\PYG{l+m+mf}{1.23}\PYG{p}{,} \PYG{l+m+mf}{3.45}\PYG{p}{,} \PYG{l+m+mf}{5.62}\PYG{p}{]}\PYG{p}{)}\PYG{p}{,} \PYG{n}{cov\PYGZus{}mats}\PYG{o}{=}\PYG{p}{(}\PYG{n}{my\PYGZus{}cov\PYGZus{}mat\PYGZus{}x}\PYG{p}{,} \PYG{n}{my\PYGZus{}cov\PYGZus{}mat\PYGZus{}y}\PYG{p}{)}\PYG{p}{)}
\end{Verbatim}

This keyword argument can be omitted, in which case covariance matrices of zero are assumed.
To specify a covariance matrix for a single axis, replace the other with \code{None}.

\begin{Verbatim}[commandchars=\\\{\}]
\PYG{g+gp}{\textgreater{}\textgreater{}\textgreater{} }\PYG{n}{my\PYGZus{}dataset} \PYG{o}{=} \PYG{n}{Dataset}\PYG{p}{(}\PYG{n}{data}\PYG{o}{=}\PYG{p}{(}\PYG{p}{[}\PYG{l+m+mf}{0.}\PYG{p}{,} \PYG{l+m+mf}{1.}\PYG{p}{,} \PYG{l+m+mf}{2.}\PYG{p}{]}\PYG{p}{,} \PYG{p}{[}\PYG{l+m+mf}{1.23}\PYG{p}{,} \PYG{l+m+mf}{3.45}\PYG{p}{,} \PYG{l+m+mf}{5.62}\PYG{p}{]}\PYG{p}{)}\PYG{p}{,} \PYG{n}{cov\PYGZus{}mats}\PYG{o}{=}\PYG{p}{(}\PYG{n+nb+bp}{None}\PYG{p}{,} \PYG{n}{my\PYGZus{}cov\PYGZus{}mat\PYGZus{}y}\PYG{p}{)}\PYG{p}{)}
\end{Verbatim}
\end{quote}

\textbf{title} : string
\begin{quote}

the name of the \emph{Dataset}. If omitted, the \emph{Dataset} will be given the generic name `Untitled Dataset'.
\end{quote}
\index{axis\_labels (Dataset attribute)}

\begin{fulllineitems}
\phantomsection\label{index:kafe.dataset.Dataset.axis_labels}\pysigline{\bfcode{axis\_labels}\strong{ = None}}
axis labels

\end{fulllineitems}

\index{axis\_units (Dataset attribute)}

\begin{fulllineitems}
\phantomsection\label{index:kafe.dataset.Dataset.axis_units}\pysigline{\bfcode{axis\_units}\strong{ = None}}
units to assume for axis

\end{fulllineitems}

\index{cov\_mat\_is\_regular() (Dataset method)}

\begin{fulllineitems}
\phantomsection\label{index:kafe.dataset.Dataset.cov_mat_is_regular}\pysiglinewithargsret{\bfcode{cov\_mat\_is\_regular}}{\emph{axis}}{}
Returns \emph{True} if the covariance matrix for an axis is regular and \code{False} if it is
singular.
\begin{description}
\item[{\textbf{axis}}] \leavevmode{[}\code{'x'} or \code{'y'}{]}
Axis for which to check for regularity of the covariance matrix.

\end{description}

\end{fulllineitems}

\index{cov\_mats (Dataset attribute)}

\begin{fulllineitems}
\phantomsection\label{index:kafe.dataset.Dataset.cov_mats}\pysigline{\bfcode{cov\_mats}\strong{ = None}}
list of covariance matrices

\end{fulllineitems}

\index{data (Dataset attribute)}

\begin{fulllineitems}
\phantomsection\label{index:kafe.dataset.Dataset.data}\pysigline{\bfcode{data}\strong{ = None}}
list containing measurement data (axis-ordering)

\end{fulllineitems}

\index{get\_axis() (Dataset method)}

\begin{fulllineitems}
\phantomsection\label{index:kafe.dataset.Dataset.get_axis}\pysiglinewithargsret{\bfcode{get\_axis}}{\emph{axis\_alias}}{}
Get axis id from an alias.
\begin{description}
\item[{\textbf{axis\_alias}}] \leavevmode{[}string or int{]}
Alias of the axis whose id should be returned. This is for example either \code{'0'} or \code{'x'} for the \emph{x}-axis (id 0).

\end{description}

\end{fulllineitems}

\index{get\_cov\_mat() (Dataset method)}

\begin{fulllineitems}
\phantomsection\label{index:kafe.dataset.Dataset.get_cov_mat}\pysiglinewithargsret{\bfcode{get\_cov\_mat}}{\emph{axis}, \emph{fallback\_on\_singular=None}}{}
Get the error matrix for an axis.
\begin{description}
\item[{\textbf{axis}}] \leavevmode{[}\code{'x'} or \code{'y'}{]}
Axis for which to load the error matrix.

\item[{\emph{fallback\_on\_singular}}] \leavevmode{[}\emph{numpy.matrix} or string (optional){]}
What to return if the matrix is singular. If this is \code{None} (default), the matrix is returned anyway.
If this is a \emph{numpy.matrix} object or similar, that is returned istead. Alternatively, the shortcuts
\code{'identity'} or \code{1} and \code{'zero'} or \code{0} can be used to return the identity and zero matrix
respectively.

\end{description}

\end{fulllineitems}

\index{get\_data() (Dataset method)}

\begin{fulllineitems}
\phantomsection\label{index:kafe.dataset.Dataset.get_data}\pysiglinewithargsret{\bfcode{get\_data}}{\emph{axis}}{}
Get the measurement data for an axis.
\begin{description}
\item[{\textbf{axis}}] \leavevmode{[}string{]}
Axis for which to get the measurement data. Can be \code{'x'} or \code{'y'}.

\end{description}

\end{fulllineitems}

\index{get\_data\_span() (Dataset method)}

\begin{fulllineitems}
\phantomsection\label{index:kafe.dataset.Dataset.get_data_span}\pysiglinewithargsret{\bfcode{get\_data\_span}}{\emph{axis}, \emph{include\_error\_bars=False}}{}
Get the data span for an axis. The data span is a tuple (\emph{min}, \emph{max}) containing
the smallest and highest coordinates for an axis.
\begin{description}
\item[{\textbf{axis}}] \leavevmode{[}\code{'x'} or \code{'y'}{]}
Axis for which to get the data span.

\item[{\emph{include\_error\_bars}}] \leavevmode{[}boolean (optional){]}
\code{True} if the returned span should be enlarged to
contain the error bars of the smallest and largest datapoints (default: \code{False})

\end{description}

\end{fulllineitems}

\index{get\_formatted() (Dataset method)}

\begin{fulllineitems}
\phantomsection\label{index:kafe.dataset.Dataset.get_formatted}\pysiglinewithargsret{\bfcode{get\_formatted}}{\emph{format\_string='.06e'}, \emph{delimiter='t'}}{}
Returns the dataset in a plain-text format which is human-readable and
can later be used as an input file for the creation of a new \emph{Dataset}.
\phantomsection\label{index:get-formatted}
The format is as follows:

\begin{Verbatim}[commandchars=\\\{\}]
\# x data
x\_1  sigma\_x\_1  
x\_2  sigma\_x\_2  cor\_x\_12
...  ...        ...       ...
x\_N  sigma\_x\_N  cor\_x\_1N  ...  cor\_x\_NN

\# y data
y\_1  sigma\_y\_1  
y\_2  sigma\_y\_2  cor\_y\_12
...  ...        ...       ...
y\_N  sigma\_y\_N  cor\_y\_1N  ...  cor\_y\_NN
\end{Verbatim}

Here, the \code{x\_i} and \code{y\_i} represent the measurement data, the \code{sigma\_?\_i} are the
statistical uncertainties of each data point, and the \code{cor\_?\_ij} are the correlation
coefficients between the \emph{i}-th and \emph{j}-th data point.

If the \code{x} or \code{y} errors are not correlated, then the entire correlation coefficient matrix
can be omitted. If there are no statistical uncertainties for an axis, the second
column can also be omitted. A blank line is required at the end of each data block!
\begin{description}
\item[{\emph{format\_string}}] \leavevmode{[}string (optional){]}
A format string with which each entry will be rendered. Default is \code{'.06e'}, which means
the numbers are represented in scientific notation with six significant digits.

\item[{\emph{delimiter}}] \leavevmode{[}string (optional){]}
A delimiter used to separate columns in the output.

\end{description}

\end{fulllineitems}

\index{get\_size() (Dataset method)}

\begin{fulllineitems}
\phantomsection\label{index:kafe.dataset.Dataset.get_size}\pysiglinewithargsret{\bfcode{get\_size}}{}{}
Get the size of the \emph{Dataset}. This is equivalent to the length of the \emph{x}-axis data.

\end{fulllineitems}

\index{has\_correlations() (Dataset method)}

\begin{fulllineitems}
\phantomsection\label{index:kafe.dataset.Dataset.has_correlations}\pysiglinewithargsret{\bfcode{has\_correlations}}{\emph{axis}}{}
Returns \emph{True} if the specified axis has correlation data, \code{False} if not.
singular.
\begin{description}
\item[{\textbf{axis}}] \leavevmode{[}\code{'x'} or \code{'y'}{]}
Axis for which to check for correlations.

\end{description}

\end{fulllineitems}

\index{has\_errors() (Dataset method)}

\begin{fulllineitems}
\phantomsection\label{index:kafe.dataset.Dataset.has_errors}\pysiglinewithargsret{\bfcode{has\_errors}}{\emph{axis}}{}
Returns \emph{True} if the specified axis has statistical error data.
\begin{description}
\item[{\textbf{axis}}] \leavevmode{[}\code{'x'} or \code{'y'}{]}
Axis for which to check for error data.

\end{description}

\end{fulllineitems}

\index{n\_axes (Dataset attribute)}

\begin{fulllineitems}
\phantomsection\label{index:kafe.dataset.Dataset.n_axes}\pysigline{\bfcode{n\_axes}\strong{ = None}}
dimensionality of the \emph{Dataset}. Currently, only 2D \emph{Datasets} are supported

\end{fulllineitems}

\index{n\_datapoints (Dataset attribute)}

\begin{fulllineitems}
\phantomsection\label{index:kafe.dataset.Dataset.n_datapoints}\pysigline{\bfcode{n\_datapoints}\strong{ = None}}
number of data points in the \emph{Dataset}

\end{fulllineitems}

\index{read\_from\_file() (Dataset method)}

\begin{fulllineitems}
\phantomsection\label{index:kafe.dataset.Dataset.read_from_file}\pysiglinewithargsret{\bfcode{read\_from\_file}}{\emph{input\_file}}{}
Reads the \emph{Dataset} object from a file.
\begin{description}
\item[{returns}] \leavevmode{[}boolean{]}
\code{True} if the read succeeded, \code{False} if not.

\end{description}

\end{fulllineitems}

\index{set\_cov\_mat() (Dataset method)}

\begin{fulllineitems}
\phantomsection\label{index:kafe.dataset.Dataset.set_cov_mat}\pysiglinewithargsret{\bfcode{set\_cov\_mat}}{\emph{axis}, \emph{mat}}{}
Set the error matrix for an axis.
\begin{description}
\item[{\textbf{axis}}] \leavevmode{[}\code{'x'} or \code{'y'}{]}
Axis for which to load the error matrix.

\item[{\textbf{mat}}] \leavevmode{[}\emph{numpy.matrix} or \code{None}{]}
Error matrix for the axis. Passing \code{None} unsets the error matrix.

\end{description}

\end{fulllineitems}

\index{set\_data() (Dataset method)}

\begin{fulllineitems}
\phantomsection\label{index:kafe.dataset.Dataset.set_data}\pysiglinewithargsret{\bfcode{set\_data}}{\emph{axis}, \emph{data}}{}
Set the measurement data for an axis.
\begin{description}
\item[{\textbf{axis}}] \leavevmode{[}\code{'x'} or \code{'y'}{]}
Axis for which to set the measurement data.

\item[{\textbf{data}}] \leavevmode{[}iterable{]}
Measurement data for axis.

\end{description}

\end{fulllineitems}

\index{write\_formatted() (Dataset method)}

\begin{fulllineitems}
\phantomsection\label{index:kafe.dataset.Dataset.write_formatted}\pysiglinewithargsret{\bfcode{write\_formatted}}{\emph{file\_path}, \emph{format\_string='.06e'}, \emph{delimiter='t'}}{}
Writes the dataset to a plain-text file. For details on the format, see {\hyperref[index:get-formatted]{get\_formatted}}.
\begin{description}
\item[{\textbf{file\_path}}] \leavevmode{[}string{]}
Path of the file object to write. \textbf{WARNING}: \emph{overwrites existing files}!

\item[{\emph{format\_string}}] \leavevmode{[}string (optional){]}
A format string with which each entry will be rendered. Default is \code{'.06e'}, which means
the numbers are represented in scientific notation with six significant digits.

\item[{\emph{delimiter}}] \leavevmode{[}string (optional){]}
A delimiter used to separate columns in the output.

\end{description}

\end{fulllineitems}


\end{fulllineitems}

\index{build\_dataset() (in module kafe.dataset)}

\begin{fulllineitems}
\phantomsection\label{index:kafe.dataset.build_dataset}\pysiglinewithargsret{\bfcode{build\_dataset}}{\emph{xdata}, \emph{ydata}, \emph{**kwargs}}{}
This helper function creates a \emph{Dataset} from a series of keyword arguments.

Valid keyword arguments are:
\begin{description}
\item[{\textbf{xdata} and \textbf{ydata}}] \leavevmode{[}list/tuple/\emph{np.array} of floats{]}
These keyword arguments are mandatory and should be iterables containing the measurement data.

\item[{\emph{error specification keywords}}] \leavevmode{[}iterable or numeric (see below){]}
A valid keyword is composed of an axis (\emph{x} or \emph{y}), an error relativity specification (\emph{abs} or \emph{rel})
and error correlation type (\emph{stat} or \emph{syst}). The errors are then set as follows:
\begin{enumerate}
\item {} \begin{description}
\item[{For statistical errors:}] \leavevmode\begin{itemize}
\item {} 
if keyword argument is iterable, the error list is set to that

\item {} 
if keyword argument is a number, an error list with identical entries is generated

\end{itemize}

\end{description}

\item {} \begin{description}
\item[{For systematic errors:}] \leavevmode\begin{itemize}
\item {} 
keyword argument \emph{must} be a single number. The global correlated error for the axis is then set to that.

\end{itemize}

\end{description}

\end{enumerate}

So, for example:

\begin{Verbatim}[commandchars=\\\{\}]
\PYG{g+gp}{\textgreater{}\textgreater{}\textgreater{} }\PYG{n}{myDataset} \PYG{o}{=} \PYG{n}{build\PYGZus{}dataset}\PYG{p}{(}\PYG{o}{.}\PYG{o}{.}\PYG{o}{.}\PYG{p}{,} \PYG{n}{yabsstat}\PYG{o}{=}\PYG{l+m+mf}{0.3}\PYG{p}{,} \PYG{n}{yrelsyst}\PYG{o}{=}\PYG{l+m+mf}{0.1}\PYG{p}{)}
\end{Verbatim}

creates a dataset where the statistical error of each \emph{y} coordinate is set to 0.3 and the overall systematic
error of \emph{y} is set to 0.1.

\end{description}

\end{fulllineitems}

\index{debug\_print() (in module kafe.dataset)}

\begin{fulllineitems}
\phantomsection\label{index:kafe.dataset.debug_print}\pysiglinewithargsret{\bfcode{debug\_print}}{\emph{message}}{}
\end{fulllineitems}



\subsection{\texttt{file\_tools} Module}
\label{index:file-tools-module}\label{index:module-kafe.file_tools}\index{kafe.file\_tools (module)}\phantomsection\label{index:module-file_tools}\index{file\_tools (module)}\index{parse\_column\_data() (in module kafe.file\_tools)}

\begin{fulllineitems}
\phantomsection\label{index:kafe.file_tools.parse_column_data}\pysiglinewithargsret{\bfcode{parse\_column\_data}}{\emph{file\_to\_parse}, \emph{field\_order='x}, \emph{y'}, \emph{delimiter=' `}, \emph{cov\_mat\_files=None}, \emph{title='Untitled Dataset'}}{}
Parses a file which contains measurement data in a one-measurement-per-row format.
The field (column) order can be specified. It defaults to \emph{x,y'. Valid field names are
{}`x}, \emph{y}, \emph{xabsstat}, \emph{yabsstat}, \emph{xrelstat}, \emph{yrelstat}. Another
valid field name is \emph{ignore} which can be used to skip a field.

Every valid measurement data file \emph{must} have an \emph{x} and a \emph{y} field.

Additionally, a delimiter can be specified. If this is a whitespace character or omitted, any
sequence of whitespace characters is assumed to separate the data.

If the measurement errors and correlations are given as covariance matrices (in a separate file),
these files can be specified using the \emph{cov\_mat\_files} argument.
\begin{description}
\item[{\textbf{file\_to\_parse}}] \leavevmode{[}file-like object or string containing a file path{]}
The file to parse.

\item[{\emph{field\_order}}] \leavevmode{[}string (optional) {]}
A string of comma-separated field names giving the order of the columns in the file. Defaults to \code{'x,y'}.

\item[{\emph{delimiter}}] \leavevmode{[}string (optional){]}
The field delimiter used in the file. Defaults to any whitespace.

\item[{\emph{cov\_mat\_files}}] \leavevmode{[}\code{None} or tuple of strings/file-like objects (optional){]}
Files which contain x- and y-covariance matrices, in that order. Defaults to \code{None}.

\item[{\textbf{return}}] \leavevmode{[}\emph{Dataset}{]}
A Dataset built from the parsed file.

\end{description}

\end{fulllineitems}

\index{parse\_matrix\_file() (in module kafe.file\_tools)}

\begin{fulllineitems}
\phantomsection\label{index:kafe.file_tools.parse_matrix_file}\pysiglinewithargsret{\bfcode{parse\_matrix\_file}}{\emph{file\_like}, \emph{delimiter=None}}{}
Read a matrix from a matrix file. The format of the matrix file should be:

\begin{Verbatim}[commandchars=\\\{\}]
\# comment row
a\_11  a\_12  ...  a\_1M
a\_21  a\_22  ...  a\_2M
...   ...   ...  ...
a\_N1  a\_N2  ...  a\_NM
\end{Verbatim}
\begin{description}
\item[{\textbf{file\_like}}] \leavevmode{[}string or file-like object{]}
File path or file object to read matrix from.

\item[{\emph{delimiter}}] \leavevmode{[}\code{None} or string (optional){]}
Column delimiter use in the matrix file. Defaults to \code{None}, meaning any whitespace.

\end{description}

\end{fulllineitems}



\subsection{\texttt{fit} Module}
\label{index:fit-module}\label{index:module-kafe.fit}\index{kafe.fit (module)}\phantomsection\label{index:module-fit}\index{fit (module)}\index{Fit (class in kafe.fit)}

\begin{fulllineitems}
\phantomsection\label{index:kafe.fit.Fit}\pysiglinewithargsret{\strong{class }\bfcode{Fit}}{\emph{dataset}, \emph{fit\_function}, \emph{external\_fcn=\textless{}function chi2 at 0x28452a8\textgreater{}}, \emph{function\_label=None}, \emph{function\_equation=None}}{}
Object representing a fit. This object references the fitted \emph{Dataset}, the fit function and the resulting fit parameters.

Necessary arguments are a \emph{Dataset} object and a fit function (which should be fitted to the \emph{Dataset}).
Optionally, an external function \emph{FCN} (whose minima should be located to find the best fit) can be specified.
If not given, the \emph{FCN} function defaults to $\chi^2$.
\begin{description}
\item[{\textbf{dataset}}] \leavevmode{[}\emph{Dataset}{]}
A \emph{Dataset} object containing all information about the data

\item[{\textbf{fit\_function}}] \leavevmode{[}function{]}
A user-defined Python function to be fitted to the data. This function's first argument must be the 
independent variable \emph{x}. All other arguments \emph{must} be named and have default values given. These
defaults are used as a starting point for the actual minimization. For example, a simple linear function
would be defined like:

\begin{Verbatim}[commandchars=\\\{\}]
\PYG{g+gp}{\textgreater{}\textgreater{}\textgreater{} }\PYG{k}{def} \PYG{n+nf}{linear\PYGZus{}2par}\PYG{p}{(}\PYG{n}{x}\PYG{p}{,} \PYG{n}{slope}\PYG{o}{=}\PYG{l+m+mi}{1}\PYG{p}{,} \PYG{n}{y\PYGZus{}intercept}\PYG{o}{=}\PYG{l+m+mi}{0}\PYG{p}{)}\PYG{p}{:}
\PYG{g+gp}{... }    \PYG{k}{return} \PYG{n}{slope} \PYG{o}{*} \PYG{n}{x} \PYG{o}{+} \PYG{n}{y\PYGZus{}intercept}
\end{Verbatim}

Be aware that choosing sensible initial values for the parameters is often crucial for a succesful fit,
particularly for functions of many parameters.

\item[{\emph{external\_fcn}}] \leavevmode{[}function (optional){]}
An external \emph{FCN} (function to minimize). This function must have the following call signature:

\begin{Verbatim}[commandchars=\\\{\}]
\PYG{g+gp}{\textgreater{}\textgreater{}\textgreater{} }\PYG{n}{FCN}\PYG{p}{(}\PYG{n}{xdata}\PYG{p}{,} \PYG{n}{ydata}\PYG{p}{,} \PYG{n}{cov\PYGZus{}mat}\PYG{p}{,} \PYG{n}{fit\PYGZus{}function}\PYG{p}{,} \PYG{n}{param\PYGZus{}values}\PYG{p}{)}
\end{Verbatim}

It should return a float. If not specified, the default $\chi^2$ \emph{FCN} is used. This should
be sufficient for most fits.

\item[{\emph{function\_label}}] \leavevmode{[}$\LaTeX$-formatted string (optional){]}
A name/label/short description of the fit function. This appears in the legend describing the fitter curve.
If omitted, this defaults to the function's Python name.

\item[{\emph{function\_equation}}] \leavevmode{[}$\LaTeX$-formatted string (optional){]}
The fit function's equation.

\end{description}
\index{call\_external\_fcn() (Fit method)}

\begin{fulllineitems}
\phantomsection\label{index:kafe.fit.Fit.call_external_fcn}\pysiglinewithargsret{\bfcode{call\_external\_fcn}}{\emph{*param\_values}}{}
Wrapper for the external \emph{FCN}. Since the actual fit process depends on finding the right parameter
values and keeping everything else constant, we can use the \emph{Dataset} object to pass known, fixed
information to the external \emph{FCN}, varying only the parameter values.
\begin{description}
\item[{\textbf{param\_values}}] \leavevmode{[}sequence of values{]}
the parameter values at which \emph{FCN} is to be evaluated

\end{description}

\end{fulllineitems}

\index{current\_cov\_mat (Fit attribute)}

\begin{fulllineitems}
\phantomsection\label{index:kafe.fit.Fit.current_cov_mat}\pysigline{\bfcode{current\_cov\_mat}\strong{ = None}}
the current covariance matrix used for the \emph{Fit}

\end{fulllineitems}

\index{current\_param\_errors (Fit attribute)}

\begin{fulllineitems}
\phantomsection\label{index:kafe.fit.Fit.current_param_errors}\pysigline{\bfcode{current\_param\_errors}\strong{ = None}}
the current uncertainties of the parameters

\end{fulllineitems}

\index{current\_param\_values (Fit attribute)}

\begin{fulllineitems}
\phantomsection\label{index:kafe.fit.Fit.current_param_values}\pysigline{\bfcode{current\_param\_values}\strong{ = None}}
the current values of the parameters

\end{fulllineitems}

\index{dataset (Fit attribute)}

\begin{fulllineitems}
\phantomsection\label{index:kafe.fit.Fit.dataset}\pysigline{\bfcode{dataset}\strong{ = None}}
this Fit instance's child \emph{Dataset}

\end{fulllineitems}

\index{do\_fit() (Fit method)}

\begin{fulllineitems}
\phantomsection\label{index:kafe.fit.Fit.do_fit}\pysiglinewithargsret{\bfcode{do\_fit}}{\emph{quiet=False}, \emph{verbose=False}}{}
Runs the fit algorithm for this \emph{Fit} object.

First, the \emph{Dataset} is fitted considering only uncertainties in the \emph{y} direction.
If the \emph{Dataset} has no uncertainties in the \emph{y} direction, they are assumed to be 
equal to 1.0 for this preliminary fit, as there is no better information available.

Next, the fit errors in the \emph{x} direction (if they exist) are taken into account by
projecting the covariance matrix for the \emph{x} errors onto the \emph{y} covariance matrix.
This is done by taking the first derivative of the fit function in each point and
``projecting'' the \emph{x} error onto the resulting tangent to the curve.

This last step is repeater until the change in the error matrix caused by the projection
becomes negligible.
\begin{description}
\item[{\emph{quiet}}] \leavevmode{[}boolean (optional){]}
Set to \code{True} if no output should be printed.

\item[{\emph{verbose}}] \leavevmode{[}boolean (optional){]}
Set to \code{True} if more output should be printed.

\end{description}

\end{fulllineitems}

\index{external\_fcn (Fit attribute)}

\begin{fulllineitems}
\phantomsection\label{index:kafe.fit.Fit.external_fcn}\pysigline{\bfcode{external\_fcn}\strong{ = None}}
the (external) function to be minimized for this \emph{Fit}

\end{fulllineitems}

\index{fit\_function (Fit attribute)}

\begin{fulllineitems}
\phantomsection\label{index:kafe.fit.Fit.fit_function}\pysigline{\bfcode{fit\_function}\strong{ = None}}
the fit function used for this \emph{Fit}

\end{fulllineitems}

\index{fit\_one\_iteration() (Fit method)}

\begin{fulllineitems}
\phantomsection\label{index:kafe.fit.Fit.fit_one_iteration}\pysiglinewithargsret{\bfcode{fit\_one\_iteration}}{\emph{verbose=False}}{}
Instructs the minimizer to do a minimization.

\end{fulllineitems}

\index{function\_equation (Fit attribute)}

\begin{fulllineitems}
\phantomsection\label{index:kafe.fit.Fit.function_equation}\pysigline{\bfcode{function\_equation}\strong{ = None}}
$\LaTeX$ function equation

\end{fulllineitems}

\index{function\_label (Fit attribute)}

\begin{fulllineitems}
\phantomsection\label{index:kafe.fit.Fit.function_label}\pysigline{\bfcode{function\_label}\strong{ = None}}
a label to use in the legend when plotting

\end{fulllineitems}

\index{get\_current\_fit\_function() (Fit method)}

\begin{fulllineitems}
\phantomsection\label{index:kafe.fit.Fit.get_current_fit_function}\pysiglinewithargsret{\bfcode{get\_current\_fit\_function}}{}{}
This method returns a function object corresponding to the fit function
for the current parameter values. The returned function is a function of
a single variable.
\begin{description}
\item[{returns}] \leavevmode{[}function{]}
A function of a single variable corresponding to the fit function at the
current parameter values.

\end{description}

\end{fulllineitems}

\index{get\_error\_matrix() (Fit method)}

\begin{fulllineitems}
\phantomsection\label{index:kafe.fit.Fit.get_error_matrix}\pysiglinewithargsret{\bfcode{get\_error\_matrix}}{}{}
This method returns the covariance matrix of the fit parameters which is obtained
by querying the minimizer object for this fit
\begin{description}
\item[{returns}] \leavevmode{[}\emph{numpy.matrix}{]}
The covariance matrix of the parameters.

\end{description}

\end{fulllineitems}

\index{get\_parameter\_errors() (Fit method)}

\begin{fulllineitems}
\phantomsection\label{index:kafe.fit.Fit.get_parameter_errors}\pysiglinewithargsret{\bfcode{get\_parameter\_errors}}{\emph{rounding=False}}{}
Get the current parameter uncertainties from the minimizer.
\begin{description}
\item[{\emph{rounding}}] \leavevmode{[}boolean (optional){]}
Whether or not to round the returned values to significance.

\item[{returns}] \leavevmode{[}tuple{]}
A tuple of the parameter uncertainties

\end{description}

\end{fulllineitems}

\index{get\_parameter\_values() (Fit method)}

\begin{fulllineitems}
\phantomsection\label{index:kafe.fit.Fit.get_parameter_values}\pysiglinewithargsret{\bfcode{get\_parameter\_values}}{\emph{rounding=False}}{}
Get the current parameter values from the minimizer.
\begin{description}
\item[{\emph{rounding}}] \leavevmode{[}boolean (optional){]}
Whether or not to round the returned values to significance.

\item[{returns}] \leavevmode{[}tuple{]}
A tuple of the parameter values

\end{description}

\end{fulllineitems}

\index{minimizer (Fit attribute)}

\begin{fulllineitems}
\phantomsection\label{index:kafe.fit.Fit.minimizer}\pysigline{\bfcode{minimizer}\strong{ = None}}
this \emph{Fit}`s minimizer (\emph{Minuit})

\end{fulllineitems}

\index{number\_of\_parameters (Fit attribute)}

\begin{fulllineitems}
\phantomsection\label{index:kafe.fit.Fit.number_of_parameters}\pysigline{\bfcode{number\_of\_parameters}\strong{ = None}}
the number of parameters

\end{fulllineitems}

\index{param\_names (Fit attribute)}

\begin{fulllineitems}
\phantomsection\label{index:kafe.fit.Fit.param_names}\pysigline{\bfcode{param\_names}\strong{ = None}}
the names of the parameters

\end{fulllineitems}

\index{param\_names\_latex (Fit attribute)}

\begin{fulllineitems}
\phantomsection\label{index:kafe.fit.Fit.param_names_latex}\pysigline{\bfcode{param\_names\_latex}\strong{ = None}}
$\LaTeX$ parameter names

\end{fulllineitems}

\index{print\_fit\_details() (Fit method)}

\begin{fulllineitems}
\phantomsection\label{index:kafe.fit.Fit.print_fit_details}\pysiglinewithargsret{\bfcode{print\_fit\_details}}{}{}
prints some fit goodness details

\end{fulllineitems}

\index{print\_fit\_results() (Fit method)}

\begin{fulllineitems}
\phantomsection\label{index:kafe.fit.Fit.print_fit_results}\pysiglinewithargsret{\bfcode{print\_fit\_results}}{}{}
prints fit results

\end{fulllineitems}

\index{print\_rounded\_fit\_parameters() (Fit method)}

\begin{fulllineitems}
\phantomsection\label{index:kafe.fit.Fit.print_rounded_fit_parameters}\pysiglinewithargsret{\bfcode{print\_rounded\_fit\_parameters}}{}{}
prints the fit parameters

\end{fulllineitems}

\index{project\_x\_covariance\_matrix() (Fit method)}

\begin{fulllineitems}
\phantomsection\label{index:kafe.fit.Fit.project_x_covariance_matrix}\pysiglinewithargsret{\bfcode{project\_x\_covariance\_matrix}}{}{}
Project the \emph{x} errors from the \emph{x} covariance matrix onto the total matrix.

This is done elementwise, according to the formula:
\begin{gather}
\begin{split}C_{\text{tot},ij} = C_{y,ij} + C_{x,ij}  \frac{\partial f}{\partial x_i}  \frac{\partial f}{\partial x_j} \end{split}\notag
\end{gather}
\end{fulllineitems}

\index{xdata (Fit attribute)}

\begin{fulllineitems}
\phantomsection\label{index:kafe.fit.Fit.xdata}\pysigline{\bfcode{xdata}\strong{ = None}}
the \emph{x} coordinates of the data points used for this \emph{Fit}

\end{fulllineitems}

\index{ydata (Fit attribute)}

\begin{fulllineitems}
\phantomsection\label{index:kafe.fit.Fit.ydata}\pysigline{\bfcode{ydata}\strong{ = None}}
the \emph{y} coordinates of the data points used for this \emph{Fit}

\end{fulllineitems}


\end{fulllineitems}

\index{chi2() (in module kafe.fit)}

\begin{fulllineitems}
\phantomsection\label{index:kafe.fit.chi2}\pysiglinewithargsret{\bfcode{chi2}}{\emph{xdata}, \emph{ydata}, \emph{cov\_mat}, \emph{fit\_function}, \emph{param\_values}}{}
A simple $\chi^2$ implementation. Calculates $\chi^2$ according to the formula:
\begin{gather}
\begin{split}\chi^2 = \lambda^T C^{-1} \lambda\end{split}\notag
\end{gather}
Here, $\lambda$ is the residual vector $\lambda = \vec{y} - \vec{f}(\vec{x})$ and
$C$ is the covariance matrix.
\begin{description}
\item[{\textbf{xdata}}] \leavevmode{[}iterable{]}
The \emph{x} measurement data

\item[{\textbf{ydata}}] \leavevmode{[}iterable{]}
The \emph{y} measurement data

\item[{\textbf{cov\_mat}}] \leavevmode{[}\emph{numpy.matrix}{]}
The total covariance matrix

\item[{\textbf{fit\_function}}] \leavevmode{[}function{]}
The fit function $f(x)$

\item[{\textbf{param\_values}}] \leavevmode{[}list/tuple{]}
The values of the parameters at which $f(x)$ should be evaluated.

\end{description}

\end{fulllineitems}

\index{round\_to\_significance() (in module kafe.fit)}

\begin{fulllineitems}
\phantomsection\label{index:kafe.fit.round_to_significance}\pysiglinewithargsret{\bfcode{round\_to\_significance}}{\emph{value}, \emph{error}, \emph{significance=2}}{}
Rounds the error to the established number of significant digits, then rounds the value to the same
order of magnitude as the error.
\begin{description}
\item[{\textbf{value}}] \leavevmode{[}float{]}
value to round to significance

\item[{\textbf{error}}] \leavevmode{[}float{]}
uncertainty of the value

\item[{\emph{significance}}] \leavevmode{[}int (optional){]}
number of significant digits of the error to consider

\end{description}

\end{fulllineitems}



\subsection{\texttt{function\_library} Module}
\label{index:module-kafe.function_library}\label{index:function-library-module}\index{kafe.function\_library (module)}\phantomsection\label{index:module-function_library}\index{function\_library (module)}\index{constant\_1par() (in module kafe.function\_library)}

\begin{fulllineitems}
\phantomsection\label{index:kafe.function_library.constant_1par}\pysiglinewithargsret{\bfcode{constant\_1par}}{\emph{x}, \emph{constant=1.0}}{}
\end{fulllineitems}

\index{exp\_2par() (in module kafe.function\_library)}

\begin{fulllineitems}
\phantomsection\label{index:kafe.function_library.exp_2par}\pysiglinewithargsret{\bfcode{exp\_2par}}{\emph{x}, \emph{growth=1.0}, \emph{constant\_factor=0.0}}{}
\end{fulllineitems}

\index{exp\_3par() (in module kafe.function\_library)}

\begin{fulllineitems}
\phantomsection\label{index:kafe.function_library.exp_3par}\pysiglinewithargsret{\bfcode{exp\_3par}}{\emph{x}, \emph{growth=1.0}, \emph{constant\_factor=0.0}, \emph{y\_offset=0.0}}{}
\end{fulllineitems}

\index{exp\_3par2() (in module kafe.function\_library)}

\begin{fulllineitems}
\phantomsection\label{index:kafe.function_library.exp_3par2}\pysiglinewithargsret{\bfcode{exp\_3par2}}{\emph{x}, \emph{growth=1.0}, \emph{constant\_factor=0.0}, \emph{x\_offset=0.0}}{}
\end{fulllineitems}

\index{exp\_4par() (in module kafe.function\_library)}

\begin{fulllineitems}
\phantomsection\label{index:kafe.function_library.exp_4par}\pysiglinewithargsret{\bfcode{exp\_4par}}{\emph{x}, \emph{growth=1.0}, \emph{constant\_factor=0.0}, \emph{x\_offset=0.0}, \emph{y\_offset=0.0}}{}
\end{fulllineitems}

\index{gauss() (in module kafe.function\_library)}

\begin{fulllineitems}
\phantomsection\label{index:kafe.function_library.gauss}\pysiglinewithargsret{\bfcode{gauss}}{\emph{x}, \emph{mean=0.0}, \emph{sigma=1.0}, \emph{scale=1.0}}{}
\end{fulllineitems}

\index{linear\_1par() (in module kafe.function\_library)}

\begin{fulllineitems}
\phantomsection\label{index:kafe.function_library.linear_1par}\pysiglinewithargsret{\bfcode{linear\_1par}}{\emph{x}, \emph{slope=1.0}}{}
\end{fulllineitems}

\index{linear\_2par() (in module kafe.function\_library)}

\begin{fulllineitems}
\phantomsection\label{index:kafe.function_library.linear_2par}\pysiglinewithargsret{\bfcode{linear\_2par}}{\emph{x}, \emph{slope=1.0}, \emph{y\_intercept=0.0}}{}
\end{fulllineitems}

\index{linear\_2par2() (in module kafe.function\_library)}

\begin{fulllineitems}
\phantomsection\label{index:kafe.function_library.linear_2par2}\pysiglinewithargsret{\bfcode{linear\_2par2}}{\emph{x}, \emph{slope=1.0}, \emph{x\_offset=0.0}}{}
\end{fulllineitems}

\index{poisson() (in module kafe.function\_library)}

\begin{fulllineitems}
\phantomsection\label{index:kafe.function_library.poisson}\pysiglinewithargsret{\bfcode{poisson}}{\emph{x}, \emph{mean=0.0}, \emph{scale=1.0}}{}
\end{fulllineitems}

\index{poly3() (in module kafe.function\_library)}

\begin{fulllineitems}
\phantomsection\label{index:kafe.function_library.poly3}\pysiglinewithargsret{\bfcode{poly3}}{\emph{x}, \emph{coeff3=1.0}, \emph{coeff2=0.0}, \emph{coeff1=0.0}, \emph{coeff0=0.0}}{}
\end{fulllineitems}

\index{poly4() (in module kafe.function\_library)}

\begin{fulllineitems}
\phantomsection\label{index:kafe.function_library.poly4}\pysiglinewithargsret{\bfcode{poly4}}{\emph{x}, \emph{coeff4=1.0}, \emph{coeff3=0.0}, \emph{coeff2=0.0}, \emph{coeff1=0.0}, \emph{coeff0=0.0}}{}
\end{fulllineitems}

\index{poly5() (in module kafe.function\_library)}

\begin{fulllineitems}
\phantomsection\label{index:kafe.function_library.poly5}\pysiglinewithargsret{\bfcode{poly5}}{\emph{x}, \emph{coeff5=1.0}, \emph{coeff4=0.0}, \emph{coeff3=0.0}, \emph{coeff2=0.0}, \emph{coeff1=0.0}, \emph{coeff0=0.0}}{}
\end{fulllineitems}

\index{quadratic\_1par() (in module kafe.function\_library)}

\begin{fulllineitems}
\phantomsection\label{index:kafe.function_library.quadratic_1par}\pysiglinewithargsret{\bfcode{quadratic\_1par}}{\emph{x}, \emph{quad\_coeff=1.0}}{}
\end{fulllineitems}

\index{quadratic\_2par() (in module kafe.function\_library)}

\begin{fulllineitems}
\phantomsection\label{index:kafe.function_library.quadratic_2par}\pysiglinewithargsret{\bfcode{quadratic\_2par}}{\emph{x}, \emph{quad\_coeff=1.0}, \emph{constant=0.0}}{}
\end{fulllineitems}

\index{quadratic\_2par2() (in module kafe.function\_library)}

\begin{fulllineitems}
\phantomsection\label{index:kafe.function_library.quadratic_2par2}\pysiglinewithargsret{\bfcode{quadratic\_2par2}}{\emph{x}, \emph{quad\_coeff=1.0}, \emph{x\_offset=0.0}}{}
\end{fulllineitems}

\index{quadratic\_3par() (in module kafe.function\_library)}

\begin{fulllineitems}
\phantomsection\label{index:kafe.function_library.quadratic_3par}\pysiglinewithargsret{\bfcode{quadratic\_3par}}{\emph{x}, \emph{quad\_coeff=1.0}, \emph{lin\_coeff=0.0}, \emph{constant=0.0}}{}
\end{fulllineitems}

\index{quadratic\_3par2() (in module kafe.function\_library)}

\begin{fulllineitems}
\phantomsection\label{index:kafe.function_library.quadratic_3par2}\pysiglinewithargsret{\bfcode{quadratic\_3par2}}{\emph{x}, \emph{quad\_coeff=1.0}, \emph{x\_offset=0.0}, \emph{constant=0.0}}{}
\end{fulllineitems}



\subsection{\texttt{function\_tools} Module}
\label{index:module-kafe.function_tools}\label{index:function-tools-module}\index{kafe.function\_tools (module)}\phantomsection\label{index:module-function_tools}\index{function\_tools (module)}\index{derivative() (in module kafe.function\_tools)}

\begin{fulllineitems}
\phantomsection\label{index:kafe.function_tools.derivative}\pysiglinewithargsret{\bfcode{derivative}}{\emph{func}, \emph{derive\_by\_index}, \emph{variables\_tuple}, \emph{derivative\_spacing}}{}
Gives $\frac{\partial f}{\partial x_k}$ for $f = f(x_0, x_1, \ldots)$. \emph{func} is $f$, \emph{variables\_tuple} is $\{x_i\}$ and \emph{derive\_by\_index} is $k$.

\end{fulllineitems}

\index{derive\_by\_parameters() (in module kafe.function\_tools)}

\begin{fulllineitems}
\phantomsection\label{index:kafe.function_tools.derive_by_parameters}\pysiglinewithargsret{\bfcode{derive\_by\_parameters}}{\emph{func}, \emph{x\_0}, \emph{param\_list}, \emph{derivative\_spacing}}{}
Returns the gradient of \emph{func} with respect to its parameters, i.e. with respect to every variable
of \emph{func} except the first one.

\end{fulllineitems}

\index{derive\_by\_x() (in module kafe.function\_tools)}

\begin{fulllineitems}
\phantomsection\label{index:kafe.function_tools.derive_by_x}\pysiglinewithargsret{\bfcode{derive\_by\_x}}{\emph{func}, \emph{x\_0}, \emph{param\_list}, \emph{derivative\_spacing}}{}
If \emph{x\_0} is iterable, gives the array of derivatives of a function $f(x, par_1, par_2, \ldots)$
around $x = x_i$ at every $x_i$ in $\vec{x}$.
If \emph{x\_0} is not iterable, gives the derivative of a function $f(x, par_1, par_2, \ldots)$ around $x = \verb!x_0!$.

\end{fulllineitems}

\index{get\_function\_property() (in module kafe.function\_tools)}

\begin{fulllineitems}
\phantomsection\label{index:kafe.function_tools.get_function_property}\pysiglinewithargsret{\bfcode{get\_function\_property}}{\emph{func}, \emph{prop}}{}
Returns a specific property of the function. This assumes that the function is defined as

\begin{Verbatim}[commandchars=\\\{\}]
\PYG{g+gp}{\textgreater{}\textgreater{}\textgreater{} }\PYG{k}{def} \PYG{n+nf}{func}\PYG{p}{(}\PYG{n}{x}\PYG{p}{,} \PYG{n}{par1}\PYG{o}{=}\PYG{l+m+mf}{1.0}\PYG{p}{,} \PYG{n}{par2}\PYG{o}{=}\PYG{l+m+mf}{3.14}\PYG{p}{,} \PYG{n}{par3}\PYG{o}{=}\PYG{l+m+mf}{2.71}\PYG{p}{,} \PYG{o}{.}\PYG{o}{.}\PYG{o}{.}\PYG{p}{)}\PYG{p}{:} \PYG{o}{.}\PYG{o}{.}\PYG{o}{.}
\end{Verbatim}
\begin{description}
\item[{\textbf{func}}] \leavevmode{[}function{]}
A function object from which to extract the property.

\item[{\textbf{prop}}] \leavevmode{[}any of \code{'name'}, \code{'parameter names'}, \code{'parameter defaults'}, \code{'number of parameters'}{]}
A string representing a property.

\end{description}

\end{fulllineitems}

\index{outer\_product() (in module kafe.function\_tools)}

\begin{fulllineitems}
\phantomsection\label{index:kafe.function_tools.outer_product}\pysiglinewithargsret{\bfcode{outer\_product}}{\emph{input\_array}}{}
Takes a \emph{NumPy} array and returns the outer (dyadic, Kronecker) product with itself.
If \emph{input\_array} is a vector $\mathbf{x}$, this returns $\mathbf{x}\mathbf{x}^T$.

\end{fulllineitems}



\subsection{\texttt{minuit} Module}
\label{index:module-kafe.minuit}\label{index:minuit-module}\index{kafe.minuit (module)}\phantomsection\label{index:module-minuit}\index{minuit (module)}\index{D\_MATRIX\_ERROR (in module kafe.minuit)}

\begin{fulllineitems}
\phantomsection\label{index:kafe.minuit.D_MATRIX_ERROR}\pysigline{\bfcode{D\_MATRIX\_ERROR}\strong{ = \{0: `Error matrix not calculated', 1: `Error matrix approximate!', 2: `Error matrix forced positive definite!', 3: `Error matrix accurate'\}}}
Error matrix status codes

\end{fulllineitems}

\index{Minuit (class in kafe.minuit)}

\begin{fulllineitems}
\phantomsection\label{index:kafe.minuit.Minuit}\pysiglinewithargsret{\strong{class }\bfcode{Minuit}}{\emph{number\_of\_parameters}, \emph{function\_to\_minimize}, \emph{par\_names}, \emph{start\_params}, \emph{param\_errors}, \emph{quiet=True}, \emph{verbose=False}}{}
A class for communicating with ROOT's function minimizer tool Minuit.
\index{FCN\_wrapper() (Minuit method)}

\begin{fulllineitems}
\phantomsection\label{index:kafe.minuit.Minuit.FCN_wrapper}\pysiglinewithargsret{\bfcode{FCN\_wrapper}}{\emph{number\_of\_parameters}, \emph{derivatives}, \emph{f}, \emph{parameters}, \emph{internal\_flag}}{}
This is actually a function called in \emph{ROOT} and acting as a C wrapper 
for our \emph{FCN}, which is implemented in Python.

This function is called by \emph{Minuit} several times during a fit. It doesn't return
anything but modifies one of its arguments (\emph{f}). This is \emph{ugly}, but it's how \emph{ROOT}`s
\code{TMinuit} works. Its argument structure is fixed and determined by \emph{Minuit}:
\begin{description}
\item[{\textbf{number\_of\_parameters}}] \leavevmode{[}int{]}
The number of parameters of the current fit

\item[{\textbf{derivatives}}] \leavevmode{[}?? {]}
Computed gradient (??)

\item[{\textbf{f}}] \leavevmode{[}C array{]}
The desired function value is in f{[}0{]} after execution.

\item[{\textbf{parameters}}] \leavevmode{[}C array{]}
A C array of parameters. Is cast to a Python list

\item[{\textbf{internal\_flag}}] \leavevmode{[}int{]}
A flag allowing for different behaviour of the function.
Can be any integer from 1 (initial run) to 4(normal run). See \emph{Minuit}`s specification.

\end{description}

\end{fulllineitems}

\index{function\_to\_minimize (Minuit attribute)}

\begin{fulllineitems}
\phantomsection\label{index:kafe.minuit.Minuit.function_to_minimize}\pysigline{\bfcode{function\_to\_minimize}\strong{ = None}}
the actual \emph{FCN} called in \code{FCN\_wrapper}

\end{fulllineitems}

\index{get\_chi2\_probability() (Minuit method)}

\begin{fulllineitems}
\phantomsection\label{index:kafe.minuit.Minuit.get_chi2_probability}\pysiglinewithargsret{\bfcode{get\_chi2\_probability}}{\emph{n\_deg\_of\_freedom}}{}
Returns the probability that an observed $\chi^2$ exceeds
the calculated value of $\chi^2$ for this fit by chance, even for a correct model.
In other words, returns the probability that a worse fit of the model to the data exists.
If this is a small value (typically \textless{}5\%), this means the fit is pretty bad. For
values below this threshold, the model very probably does not fit the data.
\begin{description}
\item[{n\_def\_of\_freedom}] \leavevmode{[}int{]}
The number of degrees of freedom. This is typically $n_       ext{datapoints} - n_    ext{parameters}$.

\end{description}

\end{fulllineitems}

\index{get\_contour() (Minuit method)}

\begin{fulllineitems}
\phantomsection\label{index:kafe.minuit.Minuit.get_contour}\pysiglinewithargsret{\bfcode{get\_contour}}{\emph{parameter1}, \emph{parameter2}, \emph{n\_points=20}}{}
Returns a list of points (2-tuples) representing a sampling of
the $1\sigma$ contour of the TMinuit fit. The \code{FCN} has to be
minimized before calling this.
\begin{description}
\item[{\textbf{parameter1}}] \leavevmode{[}int{]}
ID of the parameter to be displayed on the \emph{x}-axis.

\item[{\textbf{parameter2}}] \leavevmode{[}int{]}
ID of the parameter to be displayed on the \emph{y}-axis.

\item[{\emph{n\_points}}] \leavevmode{[}int (optional){]}
number of points used to draw the contour. Default is 20.

\item[{\emph{returns}}] \leavevmode{[}2-tuple of tuples{]}
a 2-tuple (x, y) containing \code{n\_points+1} points sampled
along the contour. The first point is repeated at the end
of the list to generate a closed contour.

\end{description}

\end{fulllineitems}

\index{get\_error\_matrix() (Minuit method)}

\begin{fulllineitems}
\phantomsection\label{index:kafe.minuit.Minuit.get_error_matrix}\pysiglinewithargsret{\bfcode{get\_error\_matrix}}{}{}
Retrieves the parameter error matrix from TMinuit.

return : \emph{numpy.matrix}

\end{fulllineitems}

\index{get\_fit\_info() (Minuit method)}

\begin{fulllineitems}
\phantomsection\label{index:kafe.minuit.Minuit.get_fit_info}\pysiglinewithargsret{\bfcode{get\_fit\_info}}{\emph{info}}{}
Retrieves other info from \emph{Minuit}.
\begin{description}
\item[{\textbf{info}}] \leavevmode{[}string{]}\begin{description}
\item[{Information about the fit to retrieve. This can be any of the following:}] \leavevmode\begin{itemize}
\item {} 
\code{'fcn'}: \emph{FCN} value at minimum,

\item {} 
\code{'edm'}: estimated distance to minimum

\item {} 
\code{'err\_def'}: \emph{Minuit} error matrix status code

\item {} 
\code{'status\_code'}: \emph{Minuit} general status code

\end{itemize}

\end{description}

\end{description}

\end{fulllineitems}

\index{get\_parameter\_errors() (Minuit method)}

\begin{fulllineitems}
\phantomsection\label{index:kafe.minuit.Minuit.get_parameter_errors}\pysiglinewithargsret{\bfcode{get\_parameter\_errors}}{}{}
Retrieves the parameter errors from TMinuit.
\begin{description}
\item[{return}] \leavevmode{[}tuple{]}
Current \emph{Minuit} parameter errors

\end{description}

\end{fulllineitems}

\index{get\_parameter\_info() (Minuit method)}

\begin{fulllineitems}
\phantomsection\label{index:kafe.minuit.Minuit.get_parameter_info}\pysiglinewithargsret{\bfcode{get\_parameter\_info}}{}{}
Retrieves parameter information from TMinuit.
\begin{description}
\item[{return}] \leavevmode{[}list of tuples{]}
\code{(param\_name, param\_val, param\_error)}

\end{description}

\end{fulllineitems}

\index{get\_parameter\_name() (Minuit method)}

\begin{fulllineitems}
\phantomsection\label{index:kafe.minuit.Minuit.get_parameter_name}\pysiglinewithargsret{\bfcode{get\_parameter\_name}}{\emph{param\_nr}}{}
Gets the name of parameter number \code{param\_nr}
\begin{description}
\item[{\textbf{param\_nr}}] \leavevmode{[}int{]}
Number of the parameter whose name to get.

\end{description}

\end{fulllineitems}

\index{get\_parameter\_values() (Minuit method)}

\begin{fulllineitems}
\phantomsection\label{index:kafe.minuit.Minuit.get_parameter_values}\pysiglinewithargsret{\bfcode{get\_parameter\_values}}{}{}
Retrieves the parameter values from TMinuit.
\begin{description}
\item[{return}] \leavevmode{[}tuple{]}
Current \emph{Minuit} parameter values

\end{description}

\end{fulllineitems}

\index{max\_iterations (Minuit attribute)}

\begin{fulllineitems}
\phantomsection\label{index:kafe.minuit.Minuit.max_iterations}\pysigline{\bfcode{max\_iterations}\strong{ = None}}
maximum number of iterations until \code{TMinuit} gives up

\end{fulllineitems}

\index{minimize() (Minuit method)}

\begin{fulllineitems}
\phantomsection\label{index:kafe.minuit.Minuit.minimize}\pysiglinewithargsret{\bfcode{minimize}}{\emph{log\_print\_level=3}}{}
Do the minimization. This calls \emph{Minuit}`s algorithms \code{MIGRAD} for minimization
and \code{HESSE} for computing/checking the parameter error matrix.

\end{fulllineitems}

\index{number\_of\_parameters (Minuit attribute)}

\begin{fulllineitems}
\phantomsection\label{index:kafe.minuit.Minuit.number_of_parameters}\pysigline{\bfcode{number\_of\_parameters}\strong{ = None}}
number of parameters to minimize for

\end{fulllineitems}

\index{reset() (Minuit method)}

\begin{fulllineitems}
\phantomsection\label{index:kafe.minuit.Minuit.reset}\pysiglinewithargsret{\bfcode{reset}}{}{}
\end{fulllineitems}

\index{set\_err() (Minuit method)}

\begin{fulllineitems}
\phantomsection\label{index:kafe.minuit.Minuit.set_err}\pysiglinewithargsret{\bfcode{set\_err}}{\emph{up\_value=1.0}}{}
Sets the \code{UP} value for Minuit.
\begin{description}
\item[{\emph{up\_value}}] \leavevmode{[}float (optional, default: 1.0){]}
This is the value by which \emph{FCN} is expected to change.

\end{description}

\end{fulllineitems}

\index{set\_parameter\_errors() (Minuit method)}

\begin{fulllineitems}
\phantomsection\label{index:kafe.minuit.Minuit.set_parameter_errors}\pysiglinewithargsret{\bfcode{set\_parameter\_errors}}{\emph{param\_errors=None}}{}
Sets the fit parameter errors. If param\_values={}`None{}`, sets the error to 1\% of the parameter value.

\end{fulllineitems}

\index{set\_parameter\_names() (Minuit method)}

\begin{fulllineitems}
\phantomsection\label{index:kafe.minuit.Minuit.set_parameter_names}\pysiglinewithargsret{\bfcode{set\_parameter\_names}}{\emph{param\_names}}{}
Sets the fit parameters. If param\_values={}`None{}`, tries to infer defaults from the function\_to\_minimize.

\end{fulllineitems}

\index{set\_parameter\_values() (Minuit method)}

\begin{fulllineitems}
\phantomsection\label{index:kafe.minuit.Minuit.set_parameter_values}\pysiglinewithargsret{\bfcode{set\_parameter\_values}}{\emph{param\_values}}{}
Sets the fit parameters. If param\_values={}`None{}`, tries to infer defaults from the function\_to\_minimize.

\end{fulllineitems}

\index{set\_print\_level() (Minuit method)}

\begin{fulllineitems}
\phantomsection\label{index:kafe.minuit.Minuit.set_print_level}\pysiglinewithargsret{\bfcode{set\_print\_level}}{\emph{print\_level=1}}{}
Sets the print level for Minuit.
\begin{description}
\item[{\emph{print\_level}}] \leavevmode{[}int (optional, default: 1 (frugal output)){]}
Tells \code{TMinuit} how much output to generate. The higher this value, the
more output it generates.

\end{description}

\end{fulllineitems}

\index{set\_strategy() (Minuit method)}

\begin{fulllineitems}
\phantomsection\label{index:kafe.minuit.Minuit.set_strategy}\pysiglinewithargsret{\bfcode{set\_strategy}}{\emph{strategy\_id=1}}{}
Sets the strategy Minuit.
\begin{description}
\item[{\emph{strategy\_id}}] \leavevmode{[}int (optional, default: 1 (optimized)){]}
Tells \code{TMinuit} to use a certain strategy. Refer to \code{TMinuit}`s
documentation for available strategies.

\end{description}

\end{fulllineitems}

\index{tolerance (Minuit attribute)}

\begin{fulllineitems}
\phantomsection\label{index:kafe.minuit.Minuit.tolerance}\pysigline{\bfcode{tolerance}\strong{ = None}}
\code{TMinuit} tolerance

\end{fulllineitems}


\end{fulllineitems}



\subsection{\texttt{numeric\_tools} Module}
\label{index:module-kafe.numeric_tools}\label{index:numeric-tools-module}\index{kafe.numeric\_tools (module)}\phantomsection\label{index:module-numeric_tools}\index{numeric\_tools (module)}\index{cor\_to\_cov() (in module kafe.numeric\_tools)}

\begin{fulllineitems}
\phantomsection\label{index:kafe.numeric_tools.cor_to_cov}\pysiglinewithargsret{\bfcode{cor\_to\_cov}}{\emph{cor\_mat}, \emph{error\_list}}{}
Converts a correlation matrix to a covariance matrix according to the formula
\begin{gather}
\begin{split}\text{Cov}_{ij} = \text{Cor}_{ij}\, \sigma_i \, \sigma_j\end{split}\notag
\end{gather}\begin{description}
\item[{\textbf{cor\_mat}}] \leavevmode{[}\emph{numpy.matrix}{]}
The correlation matrix to convert.

\item[{\textbf{error\_list}}] \leavevmode{[}sequence of floats{]}
A sequence of statistical errors. Must be of the same length
as the diagonal of \emph{cor\_mat}.

\end{description}

\end{fulllineitems}

\index{cov\_to\_cor() (in module kafe.numeric\_tools)}

\begin{fulllineitems}
\phantomsection\label{index:kafe.numeric_tools.cov_to_cor}\pysiglinewithargsret{\bfcode{cov\_to\_cor}}{\emph{cov\_mat}}{}
Converts a covariance matrix to a correlation matrix according to the formula
\begin{gather}
\begin{split}\text{Cor}_{ij} = \frac{\text{Cov}_{ij}}{\sqrt{ \text{Cov}_{ii}\,\text{Cov}_{jj}}}\end{split}\notag
\end{gather}\begin{description}
\item[{\textbf{cov\_mat}}] \leavevmode
The covariance matrix to convert. Type: \emph{numpy.matrix}

\end{description}

\end{fulllineitems}

\index{extract\_statistical\_errors() (in module kafe.numeric\_tools)}

\begin{fulllineitems}
\phantomsection\label{index:kafe.numeric_tools.extract_statistical_errors}\pysiglinewithargsret{\bfcode{extract\_statistical\_errors}}{\emph{cov\_mat}}{}
Extracts the statistical errors from a covariance matrix. This means
it returns the (elementwise) square root of the diagonal entries
\begin{description}
\item[{\textbf{cov\_mat}}] \leavevmode
The covariance matrix to extract errors from. Type: \emph{numpy.matrix}

\end{description}

\end{fulllineitems}

\index{make\_symmetric\_lower() (in module kafe.numeric\_tools)}

\begin{fulllineitems}
\phantomsection\label{index:kafe.numeric_tools.make_symmetric_lower}\pysiglinewithargsret{\bfcode{make\_symmetric\_lower}}{\emph{mat}}{}
Copies the matrix entries below the main diagonal to the upper triangle half of
the matrix. Leaves the diagonal unchanged. Returns a \emph{NumPy} matrix object.
\begin{description}
\item[{\textbf{mat}}] \leavevmode{[}\emph{numpy.matrix}{]}
A lower diagonal matrix.

\item[{returns}] \leavevmode{[}\emph{numpy.matrix}{]}
The lower triangle matrix.

\end{description}

\end{fulllineitems}

\index{zero\_pad\_lower\_triangle() (in module kafe.numeric\_tools)}

\begin{fulllineitems}
\phantomsection\label{index:kafe.numeric_tools.zero_pad_lower_triangle}\pysiglinewithargsret{\bfcode{zero\_pad\_lower\_triangle}}{\emph{triangle\_list}}{}
Converts a list of lists into a lower triangle matrix. The list members should
be lists of increasing length from 1 to N, N being the dimension of
the resulting lower triangle matrix. Returns a \emph{NumPy} matrix object.

For example:

\begin{Verbatim}[commandchars=\\\{\}]
\PYG{g+gp}{\textgreater{}\textgreater{}\textgreater{} }\PYG{n}{zero\PYGZus{}pad\PYGZus{}lower\PYGZus{}triangle}\PYG{p}{(}\PYG{p}{[} \PYG{p}{[}\PYG{l+m+mf}{1.0}\PYG{p}{]}\PYG{p}{,} \PYG{p}{[}\PYG{l+m+mf}{0.2}\PYG{p}{,} \PYG{l+m+mf}{1.0}\PYG{p}{]}\PYG{p}{,} \PYG{p}{[}\PYG{l+m+mf}{0.01}\PYG{p}{,} \PYG{l+m+mf}{0.4}\PYG{p}{,} \PYG{l+m+mf}{3.0}\PYG{p}{]} \PYG{p}{]}\PYG{p}{)}
\PYG{g+go}{matrix([[ 1.  ,  0.  ,  0.  ],}
\PYG{g+go}{        [ 0.2 ,  1.  ,  0.  ],}
\PYG{g+go}{        [ 0.01,  0.4 ,  3.  ]])}
\end{Verbatim}
\begin{description}
\item[{\textbf{triangle\_list}}] \leavevmode{[}list{]}
A list containing lists of increasing length.

\item[{returns}] \leavevmode{[}\emph{numpy.matrix} {]}
The lower triangle matrix.

\end{description}

\end{fulllineitems}



\subsection{\texttt{plot} Module}
\label{index:module-kafe.plot}\label{index:plot-module}\index{kafe.plot (module)}\phantomsection\label{index:module-plot}\index{plot (module)}\index{Plot (class in kafe.plot)}

\begin{fulllineitems}
\phantomsection\label{index:kafe.plot.Plot}\pysiglinewithargsret{\strong{class }\bfcode{Plot}}{\emph{*fits}, \emph{**kwargs}}{}~\index{axis\_labels (Plot attribute)}

\begin{fulllineitems}
\phantomsection\label{index:kafe.plot.Plot.axis_labels}\pysigline{\bfcode{axis\_labels}\strong{ = None}}
axis labels

\end{fulllineitems}

\index{compute\_plot\_range() (Plot method)}

\begin{fulllineitems}
\phantomsection\label{index:kafe.plot.Plot.compute_plot_range}\pysiglinewithargsret{\bfcode{compute\_plot\_range}}{\emph{include\_error\_bars=True}}{}
Compute the span of all child datasets and sets the plot range to that

\end{fulllineitems}

\index{draw\_fit\_parameters\_box() (Plot method)}

\begin{fulllineitems}
\phantomsection\label{index:kafe.plot.Plot.draw_fit_parameters_box}\pysiglinewithargsret{\bfcode{draw\_fit\_parameters\_box}}{\emph{plot\_spec=0}}{}
Draw the parameter box to the canvas
\begin{description}
\item[{\emph{plot\_spec}}] \leavevmode{[}int, list of ints, string or None (optional, default: 0){]}
Specify the plot id of the plot for which to draw the parameters. Passing 0 will only draw the
parameter box for the first plot, and so on. Passing a list of ints will only draw the parameters
for plot ids inside the list. Passing \code{'all'} will print parameters for all plots. Passing
\code{None} will return immediately doing nothing.

\end{description}

\end{fulllineitems}

\index{draw\_legend() (Plot method)}

\begin{fulllineitems}
\phantomsection\label{index:kafe.plot.Plot.draw_legend}\pysiglinewithargsret{\bfcode{draw\_legend}}{}{}
Draw the plot legend to the canvas

\end{fulllineitems}

\index{extend\_span() (Plot method)}

\begin{fulllineitems}
\phantomsection\label{index:kafe.plot.Plot.extend_span}\pysiglinewithargsret{\bfcode{extend\_span}}{\emph{axis}, \emph{new\_span}}{}
Expand the span of the current plot.

This method extends the current plot span to include \emph{new\_span}

\end{fulllineitems}

\index{fits (Plot attribute)}

\begin{fulllineitems}
\phantomsection\label{index:kafe.plot.Plot.fits}\pysigline{\bfcode{fits}\strong{ = None}}
list of {\color{red}\bfseries{}{}`}Fit{}`s to plot

\end{fulllineitems}

\index{init\_plots() (Plot method)}

\begin{fulllineitems}
\phantomsection\label{index:kafe.plot.Plot.init_plots}\pysiglinewithargsret{\bfcode{init\_plots}}{}{}
Initialize the plots for each fit.

\end{fulllineitems}

\index{plot() (Plot method)}

\begin{fulllineitems}
\phantomsection\label{index:kafe.plot.Plot.plot}\pysiglinewithargsret{\bfcode{plot}}{\emph{p\_id}, \emph{show\_data=True}}{}
Plot the \emph{Fit} object with the number \emph{p\_id} to its figure.

\end{fulllineitems}

\index{plot\_all() (Plot method)}

\begin{fulllineitems}
\phantomsection\label{index:kafe.plot.Plot.plot_all}\pysiglinewithargsret{\bfcode{plot\_all}}{\emph{show\_info\_for='all'}, \emph{show\_data\_for='all'}}{}
Plot every \emph{Fit} object to its figure.

\end{fulllineitems}

\index{plot\_range (Plot attribute)}

\begin{fulllineitems}
\phantomsection\label{index:kafe.plot.Plot.plot_range}\pysigline{\bfcode{plot\_range}\strong{ = None}}
plot range

\end{fulllineitems}

\index{plot\_style (Plot attribute)}

\begin{fulllineitems}
\phantomsection\label{index:kafe.plot.Plot.plot_style}\pysigline{\bfcode{plot\_style}\strong{ = None}}
plot style

\end{fulllineitems}

\index{save() (Plot method)}

\begin{fulllineitems}
\phantomsection\label{index:kafe.plot.Plot.save}\pysiglinewithargsret{\bfcode{save}}{\emph{output\_file}}{}
Save the \emph{Plot} to a file.

\end{fulllineitems}

\index{show() (Plot method)}

\begin{fulllineitems}
\phantomsection\label{index:kafe.plot.Plot.show}\pysiglinewithargsret{\bfcode{show}}{}{}
Show the \emph{Plot} in a matplotlib interactive window.

\end{fulllineitems}

\index{show\_legend (Plot attribute)}

\begin{fulllineitems}
\phantomsection\label{index:kafe.plot.Plot.show_legend}\pysigline{\bfcode{show\_legend}\strong{ = None}}
whether to show the plot legend (\code{True}) or not (\code{False})

\end{fulllineitems}


\end{fulllineitems}

\index{PlotStyle (class in kafe.plot)}

\begin{fulllineitems}
\phantomsection\label{index:kafe.plot.PlotStyle}\pysigline{\strong{class }\bfcode{PlotStyle}}
Class for specifying a style for a specific plot. This object stores a progression of marker and line types and
colors, as well as preferences relating to point size and label size. These can be overriden
by overwriting the instance variables directly. A series of \emph{get\_...} methods are provided which
go through these lists cyclically.
\index{get\_line() (PlotStyle method)}

\begin{fulllineitems}
\phantomsection\label{index:kafe.plot.PlotStyle.get_line}\pysiglinewithargsret{\bfcode{get\_line}}{\emph{idm}}{}
Get a specific line type. This runs cyclically through the defined defaults.

\end{fulllineitems}

\index{get\_linecolor() (PlotStyle method)}

\begin{fulllineitems}
\phantomsection\label{index:kafe.plot.PlotStyle.get_linecolor}\pysiglinewithargsret{\bfcode{get\_linecolor}}{\emph{idm}}{}
Get a specific line color. This runs cyclically through the defined defaults.

\end{fulllineitems}

\index{get\_marker() (PlotStyle method)}

\begin{fulllineitems}
\phantomsection\label{index:kafe.plot.PlotStyle.get_marker}\pysiglinewithargsret{\bfcode{get\_marker}}{\emph{idm}}{}
Get a specific marker type. This runs cyclically through the defined defaults.

\end{fulllineitems}

\index{get\_markercolor() (PlotStyle method)}

\begin{fulllineitems}
\phantomsection\label{index:kafe.plot.PlotStyle.get_markercolor}\pysiglinewithargsret{\bfcode{get\_markercolor}}{\emph{idm}}{}
Get a specific marker color. This runs cyclically through the defined defaults.

\end{fulllineitems}

\index{get\_pointsize() (PlotStyle method)}

\begin{fulllineitems}
\phantomsection\label{index:kafe.plot.PlotStyle.get_pointsize}\pysiglinewithargsret{\bfcode{get\_pointsize}}{\emph{idm}}{}
Get a specific point size. This runs cyclically through the defined defaults.

\end{fulllineitems}


\end{fulllineitems}

\index{label\_to\_latex() (in module kafe.plot)}

\begin{fulllineitems}
\phantomsection\label{index:kafe.plot.label_to_latex}\pysiglinewithargsret{\bfcode{label\_to\_latex}}{\emph{label}}{}
Generates a simple LaTeX-formatted label from a plain-text label.
This treats isolated characters and words beginning with a backslash
as mathematical expressions and surround them with \$ signs accordingly.
\begin{description}
\item[{\textbf{label}}] \leavevmode{[}string{]}
Plain-text string to convert to LaTeX.

\end{description}

\end{fulllineitems}

\index{pad\_span() (in module kafe.plot)}

\begin{fulllineitems}
\phantomsection\label{index:kafe.plot.pad_span}\pysiglinewithargsret{\bfcode{pad\_span}}{\emph{span}, \emph{pad\_coeff=1}, \emph{additional\_pad=None}}{}
Enlarges the interval \emph{span} (list of two floats) symmetrically around
its center to length \emph{pad\_coeff}. Optionally, an \emph{additional\_pad} argument
can be specified. The returned span is then additionally enlarged by that amount.

\emph{additional\_pad} can also be a list of two floats which specifies an asymmetric
amount by which to enlarge the span. Note that in this case, positive entries in
\emph{additional\_pad} will enlarge the span (move the interval end away from the
interval's center) and negative amounts will shorten it (move the interval end
towards the interval's center).

\end{fulllineitems}



\subsection{\texttt{stream} Module}
\label{index:stream-module}\label{index:module-kafe.stream}\index{kafe.stream (module)}\phantomsection\label{index:module-stream}\index{stream (module)}\index{StreamDup (class in kafe.stream)}

\begin{fulllineitems}
\phantomsection\label{index:kafe.stream.StreamDup}\pysiglinewithargsret{\strong{class }\bfcode{StreamDup}}{\emph{out\_file}}{}
Bases: \code{object}

Object for simultaneous logging to stdout and a file.
\index{fileno() (StreamDup method)}

\begin{fulllineitems}
\phantomsection\label{index:kafe.stream.StreamDup.fileno}\pysiglinewithargsret{\bfcode{fileno}}{}{}
\end{fulllineitems}

\index{flush() (StreamDup method)}

\begin{fulllineitems}
\phantomsection\label{index:kafe.stream.StreamDup.flush}\pysiglinewithargsret{\bfcode{flush}}{}{}
\end{fulllineitems}

\index{write() (StreamDup method)}

\begin{fulllineitems}
\phantomsection\label{index:kafe.stream.StreamDup.write}\pysiglinewithargsret{\bfcode{write}}{\emph{message}}{}
\end{fulllineitems}

\index{write\_timestamp() (StreamDup method)}

\begin{fulllineitems}
\phantomsection\label{index:kafe.stream.StreamDup.write_timestamp}\pysiglinewithargsret{\bfcode{write\_timestamp}}{\emph{prefix}}{}
\end{fulllineitems}

\index{write\_to\_file() (StreamDup method)}

\begin{fulllineitems}
\phantomsection\label{index:kafe.stream.StreamDup.write_to_file}\pysiglinewithargsret{\bfcode{write\_to\_file}}{\emph{message}}{}
\end{fulllineitems}

\index{write\_to\_stdout() (StreamDup method)}

\begin{fulllineitems}
\phantomsection\label{index:kafe.stream.StreamDup.write_to_stdout}\pysiglinewithargsret{\bfcode{write\_to\_stdout}}{\emph{message}}{}
\end{fulllineitems}


\end{fulllineitems}



\subsection{Indices and tables}
\label{index:indices-and-tables}\begin{itemize}
\item {} 
\emph{genindex}

\item {} 
\emph{modindex}

\item {} 
\emph{search}

\end{itemize}


\renewcommand{\indexname}{Python Module Index}
\begin{theindex}
\def\bigletter#1{{\Large\sffamily#1}\nopagebreak\vspace{1mm}}
\bigletter{c}
\item {\texttt{constants}} \emph{(Unix)}, \pageref{index:module-constants}
\indexspace
\bigletter{d}
\item {\texttt{dataset}} \emph{(Unix)}, \pageref{index:module-dataset}
\indexspace
\bigletter{f}
\item {\texttt{file\_tools}} \emph{(Unix)}, \pageref{index:module-file_tools}
\item {\texttt{fit}} \emph{(Unix)}, \pageref{index:module-fit}
\item {\texttt{function\_library}} \emph{(Unix)}, \pageref{index:module-function_library}
\item {\texttt{function\_tools}} \emph{(Unix)}, \pageref{index:module-function_tools}
\indexspace
\bigletter{k}
\item {\texttt{kafe}}, \pageref{index:module-kafe}
\item {\texttt{kafe.\_\_init\_\_}}, \pageref{index:module-kafe.__init__}
\item {\texttt{kafe.\_version\_info}}, \pageref{index:module-kafe._version_info}
\item {\texttt{kafe.constants}}, \pageref{index:module-kafe.constants}
\item {\texttt{kafe.dataset}}, \pageref{index:module-kafe.dataset}
\item {\texttt{kafe.file\_tools}}, \pageref{index:module-kafe.file_tools}
\item {\texttt{kafe.fit}}, \pageref{index:module-kafe.fit}
\item {\texttt{kafe.function\_library}}, \pageref{index:module-kafe.function_library}
\item {\texttt{kafe.function\_tools}}, \pageref{index:module-kafe.function_tools}
\item {\texttt{kafe.minuit}}, \pageref{index:module-kafe.minuit}
\item {\texttt{kafe.numeric\_tools}}, \pageref{index:module-kafe.numeric_tools}
\item {\texttt{kafe.plot}}, \pageref{index:module-kafe.plot}
\item {\texttt{kafe.stream}}, \pageref{index:module-kafe.stream}
\indexspace
\bigletter{m}
\item {\texttt{minuit}} \emph{(Unix)}, \pageref{index:module-minuit}
\indexspace
\bigletter{n}
\item {\texttt{numeric\_tools}} \emph{(Unix)}, \pageref{index:module-numeric_tools}
\indexspace
\bigletter{p}
\item {\texttt{plot}} \emph{(Unix)}, \pageref{index:module-plot}
\indexspace
\bigletter{s}
\item {\texttt{stream}} \emph{(Unix)}, \pageref{index:module-stream}
\end{theindex}

\renewcommand{\indexname}{Index}
\printindex
\end{document}
